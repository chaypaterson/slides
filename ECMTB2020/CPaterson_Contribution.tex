\documentclass[a4paper]{article}
\usepackage[margin=25mm]{geometry}
\usepackage{amsfonts}
\usepackage{amssymb}
\usepackage{graphicx}
\usepackage{amsmath}

\title{Evolution on graphs and the genetics of vestibular schwannoma}
\author{Chay Paterson$^{1,2}$, Miriam J Smith$^{1}$, Ivana Bozic$^{3}$, Xanthe
Hoad$^{4}$, D Gareth R Evans$^{1}$  \\
        \small $^{1}$University of Manchester \\
        \small $^{2}$InSync Technology \\
        \small $^{3}$University of Washington \\
        \small $^{4}$University Hospital Southampton \\
}
\date{chay.paterson@manchester.ac.uk} % speaker email  

% Please do not add any other package!!!
% Please do not add any figure!!!
% If you have any question about latex template please contact us info@ecmtb2020.org

\begin{document}
\maketitle

Vestibular schwannomas are benign central nervous system tumours. When treated
with radiotherapy, the tumours may become malignant instead of being
successfully treated. By developing a mathematical model in close collaboration
with clinical geneticists, we can hypothesise that malignancy must be caused by
at least one tumour suppressor, rather than an oncogene; and can also constrain
the size and location of this unknown gene [1]. Futhermore, there should be an
interesting mathematical relation between aneuploidy in schwannomas in the
general population, and the risk of schwannoma in populations affected by a certain rare
genetic disorder [2].


{\bf References}
\newline
[1] C Paterson, MJ Smith, I Bozic, X Hoad, DGR Evans, ``A mechanistic mathematical model of initiation and malignant transformation in sporadic vestibular schwannoma'', British Journal of
Cancer, (Under review), 2022.
\newline
[2] C Paterson, MJ Smith, DGR Evans, ``Aneuploidy and relative risk of sporadic
schwannoma in 22q11.2 deletion syndrome'', Journal of Mathematical Biology,
(Submitted), 2022.


\end{document}
