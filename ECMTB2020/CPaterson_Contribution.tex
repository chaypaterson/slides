\documentclass[a4paper]{article}
\usepackage[margin=25mm]{geometry}
\usepackage{amsfonts}
\usepackage{amssymb}
\usepackage{graphicx}
\usepackage{amsmath}

\title{Where is Neptune? Evolution on graphs and vestibular schwannoma}
\author{Chay Paterson$^{1,2}$, Miriam J Smith$^{1}$, Ivana Bozic$^{3}$, Xanthe
Hoad$^{4}$, D Gareth R Evans$^{1}$  \\
        \small $^{1}$University of Manchester \\
        \small $^{2}$InSync Technology \\
        \small $^{3}$University of Washington \\
        \small $^{4}$University Hospital Southampton \\
}
\date{chay.paterson@manchester.ac.uk} % speaker email  

% Please do not add any other package!!!
% Please do not add any figure!!!
% If you have any question about latex template please contact us info@ecmtb2020.org

\begin{document}
\maketitle

Vestibular schwannomas are benign central nervous system tumours. When treated
with radiotherapy, the tumours may become malignant instead of being
successfully treated. It is not clear if radiotherapy is causative or not. By
developing mathematical models in close collaboration with clinical geneticists,
we can hypothesise that malignancy is more likely to be caused by a
loss-of-function mutation than a gain-of-function mutation. By training the model on a combination of epidemiological, genomic, and clinical data, we can also constrain
the size and location of this unknown gene, \emph{TSX} [1]. We conclude that
radiotherapy is probably safe for sporadic tumours, but the model has not yet
been extended to inherited NF2 cases.

{\bf References}
\newline
[1] C Paterson, MJ Smith, I Bozic, X Hoad, DGR Evans, ``A mechanistic mathematical model of initiation and malignant transformation in sporadic vestibular schwannoma'', Under review, 2022.

\end{document}
